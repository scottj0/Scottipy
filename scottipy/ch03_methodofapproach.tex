%
% $Id: ch03_thework.tex
%
%   *******************************************************************
%   * SEE THE MAIN FILE "AllegThesis.tex" FOR MORE INFORMATION.       *
%   *******************************************************************
%
\chapter{Method of Approach} \label{ch:method}
This chapter outlines how the process of building the project will work. It also
shows how the Spotify API works, and includes an outline of Spotipy and Flask.

\section{User Inputs}

The existing Spotify API allows for searching for tracks, albums, artists, and playlists, which
makes it easy for the user to enter a song name and quickly find it. For example, here is a query
for the term "dark fantasy":
\\

\lstset{language=Java}
\begin{lstlisting}
curl -X "GET" "https://api.spotify.com/v1/search?q=dark%20fantasy&type=track" -H "Accept: application/json"
-H "Content-Type: application/json"
\end{lstlisting}

The result is the Spotify link to the track, as well as information about the popularity,
and listing on its album.\\

\begin{lstlisting}
"href": "https://api.spotify.com/v1/tracks/7yNK27ZTpHew0c55VvIJgm",
        "id": "7yNK27ZTpHew0c55VvIJgm",
        "is_local": false,
        "name": "Dark Fantasy",
        "popularity": 66,
        "preview_url": null,
        "track_number": 1,
        "type": "track",
        "uri": "spotify:track:7yNK27ZTpHew0c55VvIJgm"
\end{lstlisting}

From the previous step, information about the song can be found by searching for its audio features:
\\
\begin{lstlisting}
curl -X "GET" "https://api.spotify.com/v1/audio-features/7yNK27ZTpHew0c55VvIJgm"
-H "Accept: application/json" -H "Content-Type: application/json"
\end{lstlisting}


This query results in an easy to use list of characteristics and their matching values:
\\

\begin{lstlisting}
{
  "danceability": 0.59,
  "energy": 0.587,
  "key": 5,
  "loudness": -5.919,
  "mode": 1,
  "speechiness": 0.0457,
  "acousticness": 0.274,
  "instrumentalness": 0,
  "liveness": 0.167,
  "valence": 0.367,
  "tempo": 88.015,
  "type": "audio_features",
  "id": "7yNK27ZTpHew0c55VvIJgm",
  "uri": "spotify:track:7yNK27ZTpHew0c55VvIJgm",
  "track_href": "https://api.spotify.com/v1/tracks/7yNK27ZTpHew0c55VvIJgm",
  "analysis_url": "https://api.spotify.com/v1/audio-analysis/7yNK27ZTpHew0c55VvIJgm",
  "duration_ms": 280787,
  "time_signature": 4
}
\end{lstlisting}

Each of the listed features can be compared to the rest of the Spotify library
to find similar tracks. In addition to tracks, keywords such as moods or events
would also be accepted as inputs. Instead of searching for tracks in that case, the
Spotify API allows us to find popular playlists with the keywords in the title.
In this example, the term "sleep" is queried, with no other context:
//

\begin{lstlisting}
curl -X "GET" "https://api.spotify.com/v1/search?q=sleep&type=playlist" -H "Accept:
application/json" -H "Content-Type: application/json"
\end{lstlisting}

This search finds the playlist called "Sleep", curated by Spotify, as the first result:
\\
\begin{lstlisting}
{
  "playlists": {
    "href": "https://api.spotify.com/v1/search?query=sleep&type=playlist&market=US&offset=0&limit=20",
    "items": [
      {
        "collaborative": false,
        "external_urls": {
          "spotify": "https://open.spotify.com/playlist/37i9dQZF1DWZd79rJ6a7lp"
        },
        "href": "https://api.spotify.com/v1/playlists/37i9dQZF1DWZd79rJ6a7lp",
        "id": "37i9dQZF1DWZd79rJ6a7lp",
        "images": [
          {
            "height": 300,
            "url": "https://i.scdn.co/image/f499b0497289a6432dcccfb6de5f57164739d525",
            "width": 300
          }
        ],
        "name": "Sleep",
        "owner": {
          "display_name": "Spotify",
          "external_urls": {
            "spotify": "https://open.spotify.com/user/spotify"
          },
          "href": "https://api.spotify.com/v1/users/spotify",
          "id": "spotify",
          "type": "user",
          "uri": "spotify:user:spotify"
        },
        "primary_color": null,
        "public": null,
        "snapshot_id": "MTU0MDkxNDI1OCwwMDAwMDBhNDAwMDAw
                    MTY2YzExYTAyNTcwMDAwMDE2MmYyYjBlOGQ4",
        "tracks": {
          "href": "https://api.spotify.com/v1/playlists/37i9dQZF1DWZd79rJ6a7lp/tracks",
          "total": 152
        },
        "type": "playlist",
        "uri": "spotify:user:spotify:playlist:37i9dQZF1DWZd79rJ6a7lp"
      },
\end{lstlisting}

Popular tracks from those would be added to the playlist created by this application.
Thus, the basis of the model can find related tracks from a user entering either a
song name, or even just an emotion that they would like to relate music to.

\section{Spotipy}

Spotipy is a \textit{Python} wrapper for the Spotify API, developed by The Echo Nest
team \cite{Lamere:17}. Spotipy gives full access to the data available from the
Spotify platform. Spotify's native resource identifiers, URL links, and IDs for
artists, tracks, playlists, and albums are all supported. Spotipy allows users to
quickly gather information such as recently played songs, saved tracks, playlists,
and albums, top played artists and songs, and even currently playing tracks.
There are also methods for featured playlists across Spotify, new releases, and
even recommendations from seed artists, genres, or tracks.

The track end points are listed below, with their types, value ranges, and descriptions.
\begin{itemize}
  \item Acousticness - float - 0.0 to 1.0 - Lack of electric sounds/instruments
  \item Danceability - float - 0.0 to 1.0 - Combination of tempo, rhythm stability,
  and beat strength
  \item Energy - float - 0.0 to 1.0 - Intensity, dynamic range, loudness, timbre, entropy
  \item Instrumentalness - float - 0.0 to 1.0 - Lack of vocals
  \item Liveness - float - 0.0 to 1.0 - If a track was performed live
  \item Loudness - float - -60 to 0 - Overall volume
  \item Speechiness - float - 0.0 to 1.0 - Presence of spoken words
  \item Tempo - float - 0 to 320 - Beats per minute of a track
  \item Valence - float - 0.0 to 1.0 - Musical positiveness of a track
\end{itemize}

Spotipy was chosen for this project simply for ease of use, and my own familiarity
with \textit{Python}, instead of using \textit{JavaScript}. This also allowed
coding to be quickly checked in the terminal, which in the testing process
was easier for me than using \textit{JavaScript}. From the beginning of the
research project to the present, I have not seen any advantage in using the native
API instead of Spotipy. From accessing the database, to general ease of use, Spotipy
had no visible limitations.

\section{Flask}
The flipside of coding in \textit{Python} rather than \textit{JavaScript}, was
the increased difficulty in creating a web app. Luckily, the Flask framework covered every need
of this project. Flask's emphasis on being a "micro framework" means that
\begin{quote}
By default, Flask does not include a database abstraction layer, form validation
or anything else where different libraries already exist that can handle that.
Instead, Flask supports extensions to add such functionality to your application
as if it was implemented in Flask itself. Numerous extensions provide database
integration, form validation, upload handling, various open authentication
technologies, and more \cite{Pallets:18}.
\end{quote}

Since the project does not currently require any advanced or complicated techniques,
a simpler framework like Flask can easily and smoothly handle the requirements.
In addition to these benefits, Flask has a strong online presence, and extensive
documentation. This became even more useful than I first realized, as it helped
me to create the HTML templates with more guidance than many other, perhaps more
robust, frameworks. The command line interface also included above average debugging,
which became quite useful in the learning process, even when my issues were in the
HTML templates, rather than the \textit{Python} source code. As this program grows
in the future, I will surely explore using other frameworks, if Flask becomes
limiting in features or speed. Django has become a likely choice, as I had the
opportunity to work with it in class this semester.
